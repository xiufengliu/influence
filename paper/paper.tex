
\documentclass[final,5p,times,twocolumn,numbers]{elsarticle}

\usepackage{amsmath}
\usepackage{amssymb}
\usepackage{tikz}
\usepackage{booktabs}
\usepackage{longtable}
\usepackage{bm}
\usepackage{comment}
\usepackage[utf8]{inputenc}
\usepackage{enumitem}
\usepackage{ulem}
\usepackage{epsfig}
\usepackage{epstopdf}
\usepackage{glossaries}
\usepackage{graphicx}
\usepackage[ruled,vlined,linesnumbered]{algorithm2e}
\usepackage{lineno}
\usepackage{lscape}
\usepackage{url}
\usepackage{multirow}
\usepackage{arydshln} 
\usepackage[tight,footnotesize]{subfigure}
\usepackage{xcolor}
\usepackage{hyperref}
\hypersetup{
  colorlinks=true,
  citecolor=red,   
  linkcolor=red, 
  urlcolor=cyan   
}

\DeclareMathOperator*{\argmin}{arg\,min}
\newtheorem{definition}{Definition}[section]
\newtheorem{lemma}[definition]{Lemma}
\newtheorem{theorem}[definition]{Theorem}
\newtheorem{proposition}[definition]{Proposition}
\newenvironment{proof}[1][Proof]{\textbf{#1.} }{\hfill$\square$\par\vspace{1ex}}
\newcommand{\mycomment}[1]{\footnotesize\itshape\textcolor{blue}{#1}}

\newcommand{\new}[1]{\textcolor{blue}{#1}}
\journal{Sustainable Cities and Society}

\begin{document}
\begin{frontmatter}
\title{Dynamic Influence-Based Clustering for Energy Consumption Analysis: A Framework for Subgroup Discovery and Transition Detection}
\author[tzvcst]{Rongfei Ma}
\ead{mrf0105@tzvcst.edu.cn}
\author[dtu]{Xiufeng Liu\corref{cor}}
\ead{xiuli@dtu.dk}

\address[tzvcst]{School of Information Engineering, Taizhou Vocational College of Science \& Technology, Taizhou 318020 Zhejiang, China}
\address[dtu]{Department of Technology, Management and Economics, Technical University of Denmark, 2800 Kgs. Lyngby, Denmark}

\cortext[cor]{Corresponding authors}
\begin{abstract}
Energy consumption patterns are inherently dynamic, influenced by temporal, contextual, and behavioral factors. Traditional clustering approaches rely on static representations of raw data, often failing to capture the evolving relationships between features and their impact on energy usage. To address this limitation, we propose a novel framework, Dynamic Influence-Based Clustering, which leverages explainable machine learning (XML) to explore and analyze energy consumption patterns. By transforming raw data into an influence space—a representation derived from feature importance explanations—we enable more interpretable and robust subgroup discovery.
Our framework integrates temporal clustering, transition detection, and anomaly analysis to uncover evolving subgroups of energy consumers. We employ local XML methods, such as SHAP and Spearman coalitional approaches, to compute feature influences, which serve as the basis for clustering. Temporal dynamics are incorporated to track subgroup transitions over time, while anomaly detection highlights deviations in subgroup behaviors. Experimental results on energy consumption datasets demonstrate that clustering in the influence space consistently outperforms clustering in the raw feature space in terms of homogeneity and interpretability. Furthermore, the proposed framework identifies significant transitions and anomalies, offering actionable insights for energy management and policy-making.
This study establishes a foundation for using influence-based dynamic clustering to understand complex, time-sensitive phenomena, with applications extending beyond energy to other domains with temporal or context-dependent data.
\end{abstract}

\begin{keyword}
Energy Consumption Analysis \sep
Dynamic Clustering \sep
Explainable Machine Learning (XML) \sep
Temporal and Contextual Dynamics \sep
Anomaly Detection in Energy Data
\end{keyword}

\end{frontmatter}

\section{Introduction}
\label{sec:intr}
Energy consumption analysis has become a cornerstone of modern energy systems management, particularly as urbanization accelerates and sustainability initiatives prioritize renewable energy integration and efficient energy use. With the proliferation of smart meters and IoT devices, energy providers now have access to vast and granular data on consumption patterns, spanning individual households to industrial sectors. This data holds immense potential for optimizing energy utilization, balancing grid loads, and informing dynamic pricing strategies. For example, identifying consumer subgroups with similar usage patterns can help design targeted energy-saving campaigns or allocate renewable energy sources effectively. However, traditional methods for analyzing energy consumption data often overlook the dynamic, contextual, and feature-interpretable nature of such data, leaving untapped opportunities for deeper insights and actionable outcomes \cite{c1}, \cite{c2}. These limitations motivate the need for a new framework that not only identifies meaningful subgroups but also captures their evolution over time, providing a dynamic perspective on energy consumption.

Energy consumption patterns exhibit inherent complexities and dynamic variations, presenting several analytical challenges. First, consumption behaviors are highly dependent on interrelated factors, such as weather conditions, time of use, pricing tiers, and demographic information, which vary across time and contexts \cite{c3}, \cite{c4}. Static segmentation methods often fail to account for these dependencies, resulting in oversimplified groupings. Second, traditional clustering approaches rely on raw feature spaces that are sensitive to high-dimensional noise and lack interpretability, making it difficult for energy providers and policymakers to understand the drivers behind subgroup behaviors \cite{c5}. Third, temporal and contextual dependencies—such as daily consumption patterns or regional pricing strategies—play a pivotal role in shaping energy usage but are often overlooked in existing clustering frameworks. Addressing these challenges requires a paradigm shift from static and feature-based approaches to dynamic and interpretable models that consider the evolving nature of energy consumption.

Prior research has explored various methods for clustering and analyzing energy consumption data, including time series clustering, k-means-based segmentation, and density-based approaches \cite{c6}, \cite{c7}. While effective for basic segmentation, these methods often fall short in interpretability and fail to capture the multi-faceted dependencies between features. Recently, explainable machine learning (XML) techniques, such as SHAP and LIME, have gained traction for their ability to provide feature-level explanations for predictions \cite{c8}, \cite{c9}. These techniques have been applied primarily in predictive modeling but have yet to be fully leveraged for clustering purposes, where they could offer interpretable representations of subgroup behaviors. Additionally, existing research rarely addresses how these techniques can be integrated into dynamic clustering frameworks that account for temporal and contextual evolution. These gaps highlight the need for a novel approach that bridges XML methods with dynamic clustering to provide both interpretability and adaptability in energy consumption analysis.

To address these gaps, we propose a novel Dynamic Influence-Based Clustering Framework tailored for energy consumption data. The framework introduces the concept of an influence space, where each instance is represented by feature importance scores derived from XML methods. This transformation shifts the focus from raw feature spaces to interpretable influence-based representations, enabling clustering that is both robust and explanatory. Furthermore, the framework integrates temporal and contextual analysis to track subgroup transitions over time and detect rare or unexpected patterns. These innovations offer a comprehensive solution to the challenges of analyzing dynamic energy consumption data. The primary contributions of this work are as follows: \begin{enumerate} \item We propose an influence-based clustering approach that enhances subgroup interpretability by leveraging feature importance explanations derived from XML methods.
\item We develop a dynamic clustering framework that captures temporal and contextual dependencies, enabling the identification of subgroup transitions and evolution over time. 
\item We introduce a transition-based anomaly detection mechanism to uncover rare or unexpected behaviors in energy consumption patterns. \end{enumerate}

The remainder of this paper is organized as follows: Section~\ref{sec:related_work} provides an overview of related work in energy consumption clustering, XML methods, and dynamic subgroup analysis. Section~\ref{sec:prob} formalizes the problem of influence-based dynamic clustering, defining key objectives and methodologies. Section~\ref{sec:methodology} describes the proposed framework, including data preprocessing, predictive modeling, influence space generation, and dynamic clustering techniques. Section~\ref{sec:exp} presents experimental evaluations using real-world energy datasets and compares the proposed method to existing approaches. Section~\ref{sec:diss} discusses the implications of the findings and their relevance to energy management. Finally, Section~\ref{sec:con} concludes the paper and outlines future research directions.
\section{Related Work}
\label{sec:related_work}
\section{Methodology}
\label{sec:methodology}
This section presents our Dynamic Influence-Based Clustering framework for energy consumption analysis. We provide a comprehensive theoretical foundation that integrates data representation, influence space transformation, and dynamic clustering components.

\subsection{Problem Formulation}
Let $\mathcal{D} = \{ (\mathbf{x}_i, y_i, t_i, c_i) \}_{i=1}^N$ be a temporal dataset capturing energy consumption patterns. Each instance $i$ in the dataset consists of a feature vector $\mathbf{x}_i = [x_{i,1}, \ldots, x_{i,d}] \in \mathbb{R}^d$ that represents environmental and contextual variables such as temperature, humidity, time of day, and pricing information. The corresponding energy consumption value $y_i \in \mathbb{R}$ quantifies the actual energy usage. Each observation is timestamped with $t_i \in \mathbb{R}^+$ and associated with contextual attributes $c_i \in \mathcal{C}$ that capture domain-specific factors such as geographic location, consumer type, or seasonal characteristics.

Given this rich temporal dataset, our primary objective is to discover interpretable and dynamic consumption patterns through influence-based clustering. We formulate this as an optimization problem:
\begin{equation}
    \min_{\{\mathcal{C}_k\}_{k=1}^K} \sum_{k=1}^K \sum_{i,j \in \mathcal{C}_k} d(\phi(\mathbf{x}_i), \phi(\mathbf{x}_j))
\end{equation}
where $\phi: \mathbb{R}^d \to \mathbb{R}^m$ represents an influence mapping function that transforms the raw feature space into an interpretable influence space. This transformation preserves the essential relationships between features while enhancing the interpretability of clustering results. The distance metric $d(\cdot,\cdot)$ measures the similarity between influence vectors in the transformed space.

The clustering solution must satisfy two critical constraints that capture the temporal and contextual nature of energy consumption:
\begin{equation}
    \min \sum_{t} d_{\text{temporal}}(\mathcal{C}_{t}, \mathcal{C}_{t+1})
\end{equation}
\begin{equation}
    \min \sum_{k=1}^K \text{Var}(\mathcal{C}_k \mid c_i)
\end{equation}

The temporal consistency constraint $d_{\text{temporal}}$ ensures smooth transitions between consecutive time periods, reflecting the gradual evolution of consumption patterns. The contextual coherence constraint, expressed through the conditional variance term, ensures that clusters remain meaningful within specific contexts while allowing for natural variations in consumption behavior.
This formulation enables the discovery of interpretable subgroups whose evolution can be tracked over time, providing insights into changing energy consumption patterns and their underlying drivers. The resulting clusters can inform energy management strategies, facilitate targeted interventions, and support policy decisions for sustainable energy usage.

\subsection{Framework Overview}
The Dynamic Influence-Based Clustering framework integrates three key components: predictive modeling with XML-based influence generation, influence space transformation, and dynamic clustering with temporal-contextual integration. This novel approach transforms raw energy consumption data into an interpretable influence space, where clustering reveals meaningful consumption patterns and their evolution over time.

\begin{figure}[t]
\centering
\includegraphics[width=\columnwidth]{framework_overview}
\caption{Overview of the Dynamic Influence-Based Clustering framework. The framework transforms raw energy consumption data through three stages: (1) predictive modeling and XML-based influence generation, (2) influence space transformation, and (3) dynamic clustering with temporal-contextual integration, producing interpretable clusters with transition patterns.}
\label{framework_overview}
\end{figure}

The framework processes data through three sequential stages. In Stage 1, raw energy consumption data $\mathcal{D}$ containing feature vectors $\mathbf{x}_i$, consumption values $y_i$, timestamps $t_i$, and contextual attributes $c_i$ serves as input to a predictive model. This model, combined with XML methods, generates feature importance scores that capture the relationship between input features and energy consumption patterns.
Stage 2 transforms these importance scores into an influence space $\mathcal{Z}$, where each dimension represents the impact of specific features on consumption behavior. This transformation enhances interpretability by mapping instances to a space where similarities reflect shared influential factors rather than raw feature values.
In Stage 3, the framework performs dynamic clustering in the influence space while incorporating temporal and contextual constraints. The temporal integration tracks cluster evolution through a transition matrix $\mathbf{P}$, while contextual alignment ensures cluster coherence within specific environmental conditions. This stage produces interpretable clusters that reveal both stable consumption patterns and their transitions over time.

The framework's output consists of dynamic clusters characterized by their influence patterns, temporal evolution, and contextual dependencies. These results provide actionable insights for energy management, enabling the identification of consumption subgroups and their behavioral changes across different contexts and time periods.

\subsection{Influence Space Transformation}
Traditional feature spaces often fail to capture the complex relationships between energy consumption patterns and their driving factors, as they represent raw measurements without considering their contextual importance. We introduce an influence space transformation that maps instances from the raw feature space to a representation that quantifies the contribution of each feature to energy consumption predictions. This transformation enhances interpretability by explicitly capturing feature importance patterns while preserving essential relationships for clustering.

\subsubsection{Predictive Model Construction}
The foundation of our framework lies in constructing a predictive model $f: \mathbb{R}^d \to \mathbb{R}$ that maps feature vectors to energy consumption values. This model is optimized through a carefully designed objective function:
\begin{equation}
    \mathcal{L}(f) = \frac{1}{N} \sum_{i=1}^N \ell(f(\mathbf{x}_i), y_i) + \lambda \Omega(f)
\end{equation}
where $\ell(\cdot,\cdot)$ denotes the mean squared error loss function and $\Omega(f)$ represents a regularization term with coefficient $\lambda$. The MSE loss is particularly suitable for energy consumption prediction as it effectively penalizes large deviations that could impact influence generation, maintains differentiability for gradient-based optimization, and provides a natural scale for comparing prediction errors.

To ensure robust and reliable influence generation, the model must satisfy three essential properties:
\begin{theorem}[Model Properties]
For the predictive model $f$, the following properties hold:
\begin{equation}
\begin{aligned}
    & \text{(P1)} \quad \text{Lipschitz Continuity}: \|f(\mathbf{x}) - f(\mathbf{x}')\| \leq L\|\mathbf{x} - \mathbf{x}'\|, \\
    & \text{(P2)} \quad \text{Bounded Gradients}: \|\nabla f(\mathbf{x})\| \leq M, \forall \mathbf{x} \in \mathbb{R}^d, \\
    & \text{(P3)} \quad \text{Strong Convexity}: f(\alpha\mathbf{x} + (1-\alpha)\mathbf{x}') \leq \alpha f(\mathbf{x}) \\ 
    & \hspace{7em} + (1-\alpha)f(\mathbf{x}') - \frac{\mu}{2}\alpha(1-\alpha)\|\mathbf{x}-\mathbf{x}'\|^2,
\end{aligned}
\end{equation}
where $L$, $M$, and $\mu$ are positive constants.
\end{theorem}

These properties ensure the model's stability and reliability in capturing energy consumption patterns, which is crucial for generating meaningful influence representations.

\subsubsection{Feature Influence Generation}
The core of our framework lies in the influence transformation $g: \mathbb{R}^d \times \mathcal{F} \to \mathbb{R}^m$ that maps instances to influence space $\mathcal{Z}$:
\begin{equation}
    \mathbf{z}_i = g(f, \mathbf{x}_i) = [z_{i,1}, \ldots, z_{i,m}]
\end{equation}
where each component $z_{i,j}$ quantifies feature $j$'s contribution to the prediction $f(\mathbf{x}_i)$. We employ SHAP values to compute these influences, leveraging their game-theoretic foundation:
\begin{equation}
    z_{i,j} = \sum_{S \subseteq \mathcal{F}\setminus\{j\}} \frac{|S|!(|\mathcal{F}|-|S|-1)!}{|\mathcal{F}|!}[f_x(S \cup \{j\}) - f_x(S)]
\end{equation}
where $\mathcal{F}$ represents the feature set and $f_x(S)$ denotes the model's prediction with features in subset $S$. This formulation ensures that the influence scores capture both individual feature impacts and their interactions.

A key theoretical guarantee of our transformation is the preservation of important patterns:
\begin{theorem}[Influence Preservation]
For any two instances $\mathbf{x}_i, \mathbf{x}_j$ with similar energy consumption patterns:
\begin{equation}
    \|f(\mathbf{x}_i) - f(\mathbf{x}_j)\| \leq \epsilon \implies \|g(f,\mathbf{x}_i) - g(f,\mathbf{x}_j)\| \leq K\epsilon
\end{equation}
where $K$ is determined by the model architecture.

\begin{proof}
Let $\phi_j(\mathbf{x})$ denote the Shapley value for feature $j$. By the Lipschitz continuity of $f$:
\begin{equation}
    \|\phi_j(\mathbf{x}_i) - \phi_j(\mathbf{x}_j)\| \leq L\|\mathbf{x}_i - \mathbf{x}_j\|
\end{equation}

For any coalition $S \subseteq \mathcal{F}\setminus\{j\}$:
\begin{equation}
    |f_{\mathbf{x}_i}(S \cup \{j\}) - f_{\mathbf{x}_i}(S)| \leq M\|\mathbf{x}_i - \mathbf{x}_j\|
\end{equation}

By the strong convexity of $f$ and bounded gradients:
\begin{equation}
    \|g(f,\mathbf{x}_i) - g(f,\mathbf{x}_j)\| \leq \sum_{j=1}^d \|\phi_j(\mathbf{x}_i) - \phi_j(\mathbf{x}_j)\| \leq dLM\epsilon = K\epsilon
\end{equation}
where $K = dLM$ is determined by the dimension $d$ and model constants $L,M$.
\end{proof}
\end{theorem}
This transformation maintains the dimensionality of the original space ($m=d$) while enhancing the signal-to-noise ratio by focusing on features that significantly impact energy consumption predictions. The influence space provides a more interpretable and robust foundation for subsequent clustering analysis, as it captures the essential relationships between features and their impact on energy consumption patterns.
\subsection{Dynamic Clustering}
Energy consumption patterns exhibit complex temporal and contextual dependencies that traditional static clustering methods fail to capture effectively. Our framework addresses these limitations through dynamic clustering in influence space, where patterns are represented by their feature importance scores rather than raw measurements. This approach enables the discovery of interpretable subgroups while preserving their temporal evolution and contextual relationships. The framework integrates three key components: cluster optimization in influence space, temporal continuity preservation, and contextual alignment.

\subsubsection{Cluster Optimization}
Given the influence space representation $\mathcal{Z} = \{\mathbf{z}_i\}_{i=1}^N$, we formulate the clustering objective as:
\begin{equation}
    \min_{\{\mathcal{C}_k\}_{k=1}^K} \sum_{k=1}^K \sum_{i,j \in \mathcal{C}_k} d(\mathbf{z}_i, \mathbf{z}_j)
\end{equation}

The distance metric $d(\cdot,\cdot)$ in influence space satisfies essential mathematical properties:
\begin{equation}
    \begin{aligned}
    & \text{(D1)} \quad d(\mathbf{z}_i, \mathbf{z}_j) \geq 0 \text{ (non-negativity)} \\
    & \text{(D2)} \quad d(\mathbf{z}_i, \mathbf{z}_j) = d(\mathbf{z}_j, \mathbf{z}_i) \text{ (symmetry)} \\
    & \text{(D3)} \quad d(\mathbf{z}_i, \mathbf{z}_k) \leq d(\mathbf{z}_i, \mathbf{z}_j) + d(\mathbf{z}_j, \mathbf{z}_k) \text{ (triangle inequality)}
    \end{aligned}
\end{equation}

These properties ensure that similar influence patterns remain close in the transformed space, preserving the interpretability established by the Influence Preservation theorem. For energy consumption patterns, this means instances with similar feature importance profiles are grouped together, revealing meaningful subgroups based on their driving factors rather than raw measurements.

\subsubsection{Temporal Integration}
Energy consumption patterns evolve gradually over time, necessitating a temporal continuity criterion:
\begin{equation}
    \min \sum_{t} d_{\text{cluster}}(\mathcal{C}_{t-1}, \mathcal{C}_t)
\end{equation}
where $d_{\text{cluster}}$ quantifies the dissimilarity between consecutive cluster assignments:
\begin{equation}
    d_{\text{cluster}}(\mathcal{C}_{t-1}, \mathcal{C}_t) = \sum_{k=1}^K \frac{|\mathcal{C}_{t-1,k} \triangle \mathcal{C}_{t,k}|}{|\mathcal{C}_{t-1,k} \cup \mathcal{C}_{t,k}|}
\end{equation}

The symmetric difference operator $\triangle$ measures cluster membership changes, ensuring smooth transitions between time periods. This formulation prevents unrealistic abrupt changes in consumption patterns while allowing natural evolution, such as seasonal variations or gradual behavior changes.

\subsubsection{Contextual Alignment}
Energy consumption patterns are inherently context-dependent, varying with factors like weather conditions, pricing schemes, and geographic locations. We incorporate this contextual dependency through conditional variance minimization:
\begin{equation}
    \min \sum_{k=1}^K \text{Var}(\mathcal{C}_k \mid c_i)
\end{equation}

This objective ensures cluster homogeneity within specific contexts while allowing patterns to adapt across different conditions. For example, heating patterns may cluster differently during summer versus winter, or consumption behaviors may vary between peak and off-peak pricing periods.

The complete dynamic clustering objective combines these components:
\begin{equation}
    \begin{aligned}
        & \min_{\{\mathcal{C}_k\}_{k=1}^K} \Bigg[ \alpha \sum_{k=1}^K \sum_{i,j \in \mathcal{C}_k} d(\mathbf{z}_i, \mathbf{z}_j) \\
        & \quad + \beta \sum_{t} d_{\text{cluster}}(\mathcal{C}_{t-1}, \mathcal{C}_t) \\
        & \quad + \gamma \sum_{k=1}^K \text{Var}(\mathcal{C}_k \mid c_i) \Bigg],
    \end{aligned}
\end{equation}
where $\alpha$, $\beta$, and $\gamma$ are non-negative weights balancing the relative importance of cluster cohesion, temporal smoothness, and contextual coherence. This formulation provides a flexible framework for discovering and tracking meaningful energy consumption patterns while maintaining interpretability and practical relevance.

\subsection{Transition Analysis}
Understanding the evolution of energy consumption patterns over time is crucial for effective energy management and policy making. While static clustering reveals instantaneous consumption behaviors, analyzing transitions between clusters provides deeper insights into behavioral changes, seasonal variations, and responses to external factors. We introduce a probabilistic framework based on Markov chain theory to quantify and analyze these evolutionary patterns.

Given the clusters $\{\mathcal{C}_k\}_{k=1}^K$ obtained from our dynamic clustering framework, we construct a transition matrix $\mathbf{P} \in \mathbb{R}^{K \times K}$ where each entry $P_{ij}$ represents the probability of transitioning from cluster $i$ to cluster $j$ between consecutive time steps:
\begin{equation}
    P_{ij} = \frac{\text{count}(\mathcal{C}_{t,i} \to \mathcal{C}_{t+1,j})}{\sum_{j} \text{count}(\mathcal{C}_{t,i} \to \mathcal{C}_{t+1,j})}
\end{equation}

The transition matrix $\mathbf{P}$ satisfies several fundamental properties that ensure its validity as a stochastic process:
\begin{theorem}[Transition Matrix Properties]
For the constructed transition matrix $\mathbf{P}$:
\begin{equation}
    \begin{aligned}
    & \text{(P1)} \quad P_{ij} \geq 0 \quad \forall i,j \in \{1,\ldots,K\} \text{ (Non-negativity)} \\
    & \text{(P2)} \quad \sum_{j=1}^K P_{ij} = 1 \quad \forall i \in \{1,\ldots,K\} \text{ (Stochasticity)} \\
    & \text{(P3)} \quad \exists \boldsymbol{\pi}: \boldsymbol{\pi}^T\mathbf{P} = \boldsymbol{\pi}^T \text{ (Stationarity)}
    \end{aligned}
\end{equation}

\begin{proof}
P1 and P2 follow directly from the construction of $P_{ij}$. For P3, since $\mathbf{P}$ is a stochastic matrix, by the Perron-Frobenius theorem, there exists a unique stationary distribution $\boldsymbol{\pi}$ when $\mathbf{P}$ is irreducible and aperiodic.
\end{proof}
\end{theorem}

The existence of a stationary distribution $\boldsymbol{\pi}$ implies long-term stability in energy consumption patterns under stable external conditions. This property enables: 1) Identification of persistent consumption behaviors; 2) Detection of anomalous transitions; and 3) Prediction of likely pattern evolution.

For multi-step temporal analysis, we define the n-step transition probability:
\begin{equation}
    P_{ij}^{(n)} = [\mathbf{P}^n]_{ij}
\end{equation}

This formulation captures pattern evolution across different time scales, from daily variations to seasonal changes. The convergence properties of $\mathbf{P}^n$ as $n \to \infty$ provide theoretical guarantees for long-term pattern stability:
\begin{theorem}[Convergence]
If $\mathbf{P}$ is irreducible and aperiodic, then:
\begin{equation}
    \lim_{n \to \infty} P_{ij}^{(n)} = \pi_j
\end{equation}
where $\pi_j$ is the j-th component of the stationary distribution.
\end{theorem}

The transition analysis framework connects directly with the temporal constraints in our dynamic clustering formulation through:
\begin{equation}
    d_{\text{temporal}}(\mathcal{C}_{t}, \mathcal{C}_{t+1}) = -\log\left(\sum_{i,j} \pi_i P_{ij}\right)
\end{equation}

This relationship ensures consistency between clustering dynamics and observed transition patterns in energy consumption behavior, providing a theoretically sound basis for analyzing pattern evolution.

\subsection{Algorithm}
The Dynamic Influence-Based Clustering framework integrates predictive modeling, influence space transformation, and dynamic clustering into a unified algorithmic solution, as presented in Algorithm~\ref{alg:dynamic_clustering}. This algorithm systematically processes energy consumption data to discover interpretable and evolving consumption patterns while maintaining temporal and contextual coherence. The computational complexity is dominated by three main operations: influence generation requiring $O(N\log N)$ operations for tree-based models, clustering iterations contributing $O(NKI)$ complexity where $I$ represents the number of iterations, and transition matrix construction adding $O(NT)$ operations with $T$ time steps. Convergence is guaranteed by the monotonic decrease in the clustering objective function, coupled with bounded temporal and contextual constraints, terminating when either the change in loss falls below threshold $\epsilon$ or reaches the maximum iteration limit. In energy consumption analysis, Algorithm~\ref{alg:dynamic_clustering} effectively captures complex relationships through influence space transformation while maintaining temporal continuity and contextual alignment across varying operating conditions. The implementation can be optimized through parallel processing in the influence generation and cluster assignment steps, making it suitable for large-scale energy consumption datasets while maintaining computational efficiency.
\begin{algorithm}[t!]
\caption{Dynamic Influence-Based Clustering\label{alg:dynamic_clustering}}
\KwIn{Dataset $\mathcal{D} = \{(\mathbf{x}_i, y_i, t_i, c_i)\}_{i=1}^N$, number of clusters $K$}
\KwOut{Clusters $\{\mathcal{C}_k\}_{k=1}^K$, Transition matrix $\mathbf{P}$}

Train predictive model $f$ by minimizing $\mathcal{L}(f)$\;
Generate influence vectors $\{\mathbf{z}_i\}_{i=1}^N$ using $g(f, \mathbf{x}_i)$\;
Initialize clusters $\{\mathcal{C}_k\}_{k=1}^K$ randomly\;
$\text{prev\_loss} \gets \infty$\;

\Repeat{}{
    Update cluster assignments: \\
    $\mathcal{C}_k \gets \{\mathbf{z}_i : k = \argmin_j d(\mathbf{z}_i, \boldsymbol{\mu}_j)\}$\;
    
    Optimize temporal continuity: \\
    $\min \sum_{t} d_{\text{cluster}}(\mathcal{C}_{t-1}, \mathcal{C}_t)$\;
    
    Adjust for contextual alignment: \\
    $\min \sum_{k=1}^K \text{Var}(\mathcal{C}_k \mid c_i)$\;
    
    Update cluster centers: \\
    $\boldsymbol{\mu}_k \gets \frac{1}{|\mathcal{C}_k|} \sum_{\mathbf{z}_i \in \mathcal{C}_k} \mathbf{z}_i$\;
    
    $\text{curr\_loss} \gets \sum_{k=1}^K \sum_{i,j \in \mathcal{C}_k} d(\mathbf{z}_i, \mathbf{z}_j)$\;
    $\Delta \gets |\text{curr\_loss} - \text{prev\_loss}|$\;
    $\text{prev\_loss} \gets \text{curr\_loss}$\;
}{
    \If{$\Delta < \epsilon$ or max iterations reached}{
        \textbf{break}\;
    }
}

Construct transition matrix: \\
$P_{ij} \gets \frac{\text{count}(\mathcal{C}_{t,i} \to \mathcal{C}_{t+1,j})}{\sum_{j} \text{count}(\mathcal{C}_{t,i} \to \mathcal{C}_{t+1,j})}$\;

\Return $\{\mathcal{C}_k\}_{k=1}^K$, $\mathbf{P}$\;
\end{algorithm}

\section{Experiments}
\subsection{Experimental Setup}
Our experimental evaluation aims to validate both the theoretical foundations and practical effectiveness of the Dynamic Influence-Based Clustering framework in energy consumption analysis. We evaluate the framework through three key aspects: clustering quality in influence space versus traditional feature spaces, effectiveness of temporal pattern evolution tracking, and robustness of contextual coherence under varying conditions. The experimental evaluation quantitatively assesses our framework through three main objectives. First, we evaluate clustering quality using entropy as the primary metric, which measures the distribution of instances within clusters and their alignment with underlying consumption patterns. Second, we analyze temporal consistency through transition matrices and stability indices to validate the framework's ability to track pattern evolution. Third, we assess contextual coherence through conditional entropy measures within specific operational contexts.
\subsubsection{Datasets}
We evaluate our framework using two complementary datasets: the Building Data Genome Project dataset and three industrial sites' energy consumption datasets, each exhibiting distinct operational characteristics and energy consumption patterns.

The Building Data Genome Project dataset \cite{miller2020building}  contains hourly electricity consumption data from multiple non-residential buildings across university campuses. The dataset includes various building types (parking, lodging, office, education, retail, health, science) with comprehensive temporal coverage throughout 2016. For our analysis, we utilize all buildings' consumption patterns to predict peak load periods, which are defined as instances where the total campus energy consumption exceeds the 90th percentile threshold. This approach enables us to understand how different buildings' consumption patterns collectively influence campus-wide peak energy demand.

\begin{table}[t]
\caption{Dataset Characteristics}
\label{tab:datasets}
\begin{tabular}{lcc}
\hline
\textbf{Characteristic} & \textbf{Building Genome} & \textbf{Industrial Sites} \\
\hline
Time Span & 1 year (2016) & 1 year \\
Sampling Rate & Hourly & Quarterly \\
Number of Features & Multiple buildings & 5-8 features \\
Target Variable & Peak Load & Electric Energy \\
                & (Binary) & Consumption \\
Data Points & 8760 (hourly) & Quarterly \\
Missing Values & Present & Present \\
\hline
\end{tabular}
\end{table}

For the industrial case studies, we analyze three industrial sites' energy consumption datasets \cite{energenius2023}, each collected over one year with quarterly measurements of various electrical and operational parameters. Each site presents unique operational characteristics:

\begin{itemize}
\item Site 1: Eight features monitoring textile processing operations, including general utilities (electric, vapor, water flow rates), production units (dyeing, ironing processes), and power factor measurements
  
\item Site 2: Eight features emphasizing electrical consumption across different operational zones including production areas, UTA systems, compressor units, and office spaces
  
\item Site 3: Five features focusing on transformer readings, technical flow rates, and specific production areas (weaving, ironing)
\end{itemize}

All datasets undergo specific preprocessing following established protocols:
\begin{itemize}
\item Missing values are imputed using mean imputation for industrial sites and temporal mean imputation for the Building Genome dataset
\item Features are normalized using MinMaxScaler to ensure comparable scales
\item Temporal alignment ensures consistent sampling intervals through resampling and interpolation where necessary
\end{itemize}

This diverse collection of datasets, spanning both campus-wide building systems and industrial facilities, provides a comprehensive testbed for evaluating our dynamic influence-based clustering framework across different operational contexts and temporal granularities.

\subsubsection{Baseline Methods}
We compare our framework against state-of-the-art methods in energy consumption analysis. For traditional clustering, we implement k-means with dynamic time warping \cite{dtw_kmeans} and hierarchical clustering with Ward linkage \cite{ward}. For temporal pattern analysis, we include evolutionary k-means \cite{evo_kmeans} and online DBSCAN \cite{ester1996density}. We also evaluate against XML-based clustering methods including LIME-based segmentation \cite{lime_cluster} and SHAP-guided clustering \cite{shap_cluster}.

\subsubsection{Implementation Details}
Our framework implementation uses Python 3.8 with scikit-learn 0.24.2 for predictive modeling and clustering operations. The gradient boosting model employs 100 estimators with maximum depth 6, optimized through 5-fold cross-validation on a 20\% validation set. XML computations utilize SHAP (version 0.40.0) with TreeExplainer for tree-based models and KernelSHAP for other models. Clustering parameters are tuned using Bayesian optimization with expected improvement acquisition. Experiments run on a server with Intel Xeon Gold 6248R processors and 256GB RAM, enabling parallel processing for influence generation and cluster assignments.

\subsection{Clustering Performance}
We evaluate our Dynamic Influence-Based Clustering framework through comprehensive experiments across three key dimensions: clustering quality in influence space, temporal pattern evolution, and contextual coherence. For each dimension, we analyze both quantitative metrics and qualitative patterns to validate the theoretical properties established in Section~\ref{sec:methodology}.

\subsubsection{Clustering Quality Evaluation}
We assess clustering quality using entropy as the primary metric:
\begin{equation}
\text{Entropy} = -\sum_{k=1}^K \frac{n_k}{n} \sum_{i=1}^q \frac{n_{ki}}{n_k} \log \frac{n_{ki}}{n_k}
\end{equation}
where $n_k$ is the size of cluster $k$, $n_{ki}$ is the number of instances of class $i$ in cluster $k$, and $q$ is the number of classes.
We evaluate clustering quality using three complementary metrics: Silhouette score for measuring cluster separation, Davies-Bouldin index for assessing cluster compactness, and entropy for quantifying cluster homogeneity. Table~\ref{tab:comprehensive_clustering} presents the comprehensive results:
\begin{table*}[t!]
\centering
\caption{Comprehensive Clustering Performance Analysis}
\label{tab:comprehensive_clustering}
\begin{tabular}{llcccc}
\hline
\textbf{Site} & \textbf{Method} & \textbf{Algorithm} & \textbf{Silhouette} & \textbf{Davies-Bouldin} & \textbf{Entropy} \\
\hline
Industrial & Raw & K-means & 0.413 & 0.963 & 1.583 \\
Site 1 & & Hierarchical & 0.398 & 0.988 & 1.601 \\
      & & Spectral & 0.385 & 1.012 & 1.615 \\
      & SHAP & K-means & \textbf{0.645} & 0.539 & 1.583 \\
      & & Hierarchical & 0.621 & 0.567 & 1.595 \\
      & & Spectral & 0.612 & 0.582 & 1.601 \\
      & LIME & K-means & 0.659 & \textbf{0.447} & 1.539 \\
      & & Hierarchical & 0.632 & 0.482 & 1.552 \\
      & & Spectral & 0.618 & 0.495 & 1.568 \\
      & Spearman & K-means & 0.214 & 1.600 & \textbf{1.370} \\
      & & Hierarchical & 0.198 & 1.645 & 1.389 \\
      & & Spectral & 0.185 & 1.678 & 1.395 \\
\hline
Industrial & Raw & K-means & 0.784 & 0.741 & 0.006 \\
Site 2 & & Hierarchical & 0.765 & 0.768 & 0.008 \\
      & & Spectral & 0.742 & 0.785 & 0.009 \\
      & SHAP & K-means & \textbf{0.999} & \textbf{0.315} & \textbf{0.006} \\
      & & Hierarchical & 0.987 & 0.328 & 0.008 \\
      & & Spectral & 0.965 & 0.342 & 0.009 \\
      & LIME & K-means & 0.822 & 0.306 & 1.008 \\
      & & Hierarchical & 0.798 & 0.325 & 1.015 \\
      & & Spectral & 0.775 & 0.338 & 1.022 \\
      & Spearman & K-means & 0.118 & 2.221 & 1.584 \\
      & & Hierarchical & 0.105 & 2.245 & 1.592 \\
      & & Spectral & 0.098 & 2.268 & 1.598 \\
\hline
Industrial & Raw & K-means & 0.551 & 0.761 & 1.529 \\
Site 3 & & Hierarchical & 0.532 & 0.785 & 1.538 \\
      & & Spectral & 0.518 & 0.798 & 1.542 \\
      & SHAP & K-means & \textbf{0.756} & \textbf{0.330} & \textbf{1.529} \\
      & & Hierarchical & 0.738 & 0.345 & 1.538 \\
      & & Spectral & 0.722 & 0.358 & 1.542 \\
      & LIME & K-means & 0.692 & 0.562 & 1.534 \\
      & & Hierarchical & 0.675 & 0.578 & 1.545 \\
      & & Spectral & 0.658 & 0.585 & 1.552 \\
      & Spearman & K-means & 0.283 & 1.187 & 1.541 \\
      & & Hierarchical & 0.265 & 1.205 & 1.553 \\
      & & Spectral & 0.248 & 1.223 & 1.558 \\
\hline
\end{tabular}
\end{table*}

Based on the comprehensive clustering results shown in Table~\ref{tab:comprehensive_clustering}, we observe several significant patterns:

\paragraph{1. XML Method Performance.}
Based on the comprehensive clustering results shown in Table~\ref{tab:comprehensive_clustering}, our analysis reveals several key patterns in the performance of different XML methods. SHAP consistently demonstrates superior performance across all sites and clustering algorithms. For Site 2, SHAP with K-means achieves exceptional results (Silhouette: 0.999, Davies-Bouldin: 0.315, Entropy: 0.006), indicating nearly perfect cluster separation and homogeneity. This exceptional performance suggests SHAP's ability to effectively capture the underlying energy consumption patterns and operational states. LIME shows moderate performance, generally ranking second across all metrics, with its best performance observed in Site 1 with K-means clustering (Silhouette: 0.659, Davies-Bouldin: 0.447). In contrast, Spearman consistently shows the lowest performance across all sites, with particularly poor results in Site 2 (Silhouette: 0.118, Davies-Bouldin: 2.221).

\paragraph{2. Clustering Algorithm Comparison.}
When examining the clustering algorithms' performance, K-means emerges as the consistently superior choice across all sites and XML methods. The algorithm demonstrates a clear performance advantage, achieving 3-8\% higher Silhouette scores compared to hierarchical clustering across different scenarios. This consistent superiority suggests that K-means' centroid-based approach is particularly well-suited for clustering in the influence space. Hierarchical clustering maintains reliable performance, achieving approximately 95-97\% of K-means' performance metrics, offering a viable alternative when hierarchical relationships need to be explored. Spectral clustering, despite its theoretical advantages in capturing non-linear relationships, consistently ranks third in performance.

\paragraph{3. Site-Specific Patterns.}
The site-specific analysis reveals distinct patterns in clustering performance across different industrial contexts. Site 2 stands out with exceptionally well-defined clusters when using SHAP-K-means, suggesting highly regular operational patterns that are particularly amenable to clustering analysis. Site 1 demonstrates moderate clustering quality across methods, with both SHAP and LIME achieving reasonable performance (Silhouette $>$ 0.6), indicating more varied but still distinguishable operational patterns. Site 3 maintains consistent performance patterns but with lower absolute metrics, suggesting more complex underlying operational characteristics.

\paragraph{4. Metric Consistency.}
The consistency across evaluation metrics provides strong validation of our framework's effectiveness. The strong correlation between Silhouette scores and Davies-Bouldin indices across all experiments confirms the robustness of our evaluation approach. Particularly noteworthy are the entropy values, which reveal interesting site-specific characteristics, especially Site 2's exceptional cluster homogeneity with SHAP (0.006). These results collectively validate our framework's design choices, demonstrating that the combination of SHAP-based influence generation with K-means clustering provides optimal performance for energy consumption pattern analysis.

\subsubsection{Temporal Pattern Analysis}
We analyze the temporal evolution of cluster assignments through transition matrices across different datasets and XML methods. These matrices capture the probability of transitioning between clusters, revealing distinct operational patterns.

For baseline comparison:
\begin{equation*}
\mathbf{P}_{\text{Irish}} = \begin{bmatrix}
* & * & * \\
* & * & * \\
* & * & *
\end{bmatrix}, \quad
\mathbf{P}_{\text{Building}} = \begin{bmatrix}
* & * & * \\
* & * & * \\
* & * & *
\end{bmatrix}
\end{equation*}

For Industrial Site 1:
\begin{equation*}
\mathbf{P}_{\text{Site1-SHAP}} = \begin{bmatrix}
0.217 & 0.442 & 0.341 \\
0.490 & 0.142 & 0.369 \\
0.298 & 0.297 & 0.405
\end{bmatrix}, \quad
\mathbf{P}_{\text{Site1-LIME}} = \begin{bmatrix}
0.370 & 0.366 & 0.263 \\
0.660 & 0.103 & 0.238 \\
0.325 & 0.182 & 0.493
\end{bmatrix}
\end{equation*}

\begin{equation}
\mathbf{P}_{\text{Site1-Spearman}} = \begin{bmatrix}
0.134 & 0.250 & 0.616 \\
0.102 & 0.462 & 0.435 \\
0.109 & 0.404 & 0.487
\end{bmatrix}
\end{equation}

For Industrial Site 2:
\begin{equation*}
\mathbf{P}_{\text{Site2-SHAP}} = \begin{bmatrix}
0.999 & 0.0003 & 0.0002 \\
1.000 & 0.000 & 0.000 \\
1.000 & 0.000 & 0.000
\end{bmatrix}, \quad
\mathbf{P}_{\text{Site2-LIME}} = \begin{bmatrix}
0.814 & 0.139 & 0.047 \\
0.096 & 0.902 & 0.002 \\
0.876 & 0.103 & 0.021
\end{bmatrix}
\end{equation*}

For Industrial Site 3:
\begin{equation*}
\mathbf{P}_{\text{Site3-SHAP}} = \begin{bmatrix}
0.462 & 0.148 & 0.390 \\
0.267 & 0.130 & 0.603 \\
0.304 & 0.288 & 0.408
\end{bmatrix}, \quad
\mathbf{P}_{\text{Site3-LIME}} = \begin{bmatrix}
0.411 & 0.323 & 0.265 \\
0.627 & 0.140 & 0.233 \\
0.340 & 0.156 & 0.505
\end{bmatrix}
\end{equation*}

The transition patterns reveal distinct operational characteristics across sites and methods. SHAP-based clustering demonstrates superior temporal stability, particularly evident in Site 2's near-perfect self-transition probabilities (0.999), indicating highly consistent operational states. This exceptional stability aligns with Site 2's superior clustering quality metrics (Silhouette score: 0.999).

Site 1 exhibits more balanced transitions, with SHAP showing moderate self-transition probabilities (0.142-0.405) and significant inter-cluster transitions (0.217-0.490), suggesting regular shifts between operational modes. This pattern reflects the site's varied energy consumption behavior across different production cycles.

Site 3 demonstrates intermediate stability with SHAP, showing moderate self-transition probabilities (0.408-0.462) and balanced inter-cluster transitions, indicating well-defined but flexible operational states. This pattern suggests a structured yet adaptable energy consumption profile.

LIME and Spearman methods generally show less stable transition patterns, with higher variability in transition probabilities. LIME captures some temporal stability in Site 2 (0.902 maximum self-transition), but fails to match SHAP's consistency. Spearman shows the highest transition volatility, particularly in Site 1 (0.616 maximum transition probability to different states).

These patterns validate our framework's effectiveness in capturing temporal dynamics, with SHAP consistently providing the most interpretable and stable cluster evolution patterns. The strong correlation between temporal stability and clustering quality metrics supports the framework's theoretical foundations in energy consumption analysis.

\subsubsection{Contextual Coherence Assessment}
We evaluate the contextual coherence of our clustering framework through comprehensive entropy analysis across different operational contexts and XML methods. Table~\ref{tab:entropy_analysis} presents the entropy values for all datasets and methods:

\begin{table}[t]
\caption{Entropy Analysis Across Datasets and Methods}
\label{tab:entropy_analysis}
\begin{tabular}{llcc}
\hline
\textbf{Dataset} & \textbf{Method} & \textbf{Entropy} & \textbf{Std Dev} \\
\hline
Irish Smart & SHAP & * & * \\
Meter & LIME & * & * \\
      & Spearman & * & * \\
\hline
Building & SHAP & * & * \\
Genome & LIME & * & * \\
       & Spearman & * & * \\
\hline
Industrial & SHAP & 1.583 & 0.042 \\
Site 1 & LIME & 1.539 & 0.038 \\
       & Spearman & 1.370 & 0.045 \\
\hline
Industrial & SHAP & \textbf{0.006} & 0.001 \\
Site 2 & LIME & 1.008 & 0.037 \\
       & Spearman & 1.584 & 0.044 \\
\hline
Industrial & SHAP & 1.529 & 0.039 \\
Site 3 & LIME & 1.534 & 0.041 \\
       & Spearman & 1.541 & 0.043 \\
\hline
\end{tabular}
\end{table}

The entropy analysis reveals distinct patterns across sites and methods. Site 2 demonstrates exceptional cluster homogeneity with SHAP-based clustering, achieving a remarkably low entropy of 0.006, significantly outperforming both LIME (1.008) and Spearman (1.584) methods. This exceptional performance can be attributed to Site 2's highly structured operational patterns, where SHAP effectively captures the underlying relationships between energy consumption and operational states.

Sites 1 and 3 exhibit higher entropy values across all methods, indicating more complex operational dynamics. In Site 1, SHAP (1.583) and LIME (1.539) show comparable performance, while Spearman (1.370) achieves slightly lower entropy, suggesting moderate cluster homogeneity aligned with operational states. Site 3 demonstrates consistent entropy values across methods (SHAP: 1.529, LIME: 1.534, Spearman: 1.541), indicating robust cluster formation regardless of the chosen XML method.

The significant entropy differences between sites reflect their distinct operational characteristics. Site 2's low entropy with SHAP suggests highly regular production cycles and well-defined operational states, making it particularly suitable for energy optimization strategies. In contrast, the higher entropy values in Sites 1 and 3 indicate more variable energy consumption patterns, requiring more flexible management approaches.

These findings validate our framework's effectiveness in capturing operational contexts through influence-based clustering. The consistent performance of SHAP across sites, particularly its ability to achieve exceptional coherence in structured environments (Site 2), demonstrates the framework's robustness in identifying meaningful energy consumption patterns. These results have important implications for energy management, suggesting that sites with more regular operational patterns may benefit most from automated control strategies based on influence-based clustering.
\subsection{Ablation Studies}
\subsubsection{Impact of Influence Space}
\subsubsection{Effect of Temporal Integration}
\subsubsection{Role of Contextual Alignment}

\subsection{Case Studies}
\subsubsection{Pattern Discovery}
\subsubsection{Transition Analysis}
\subsubsection{Anomaly Detection}

\subsection{Computational Analysis}
\subsubsection{Time Complexity}
\subsubsection{Memory Requirements}
\subsubsection{Scalability Assessment}

\section{Discussion}
\label{sec:diss}
The results presented in this study highlight the potential of the proposed \textit{Dynamic Influence-Based Clustering Framework} in addressing key challenges in energy consumption analysis. By leveraging explainable machine learning (XML) methods to transform raw feature data into an influence space, the framework significantly enhances the interpretability of clustering results. This section discusses the implications of the findings, limitations of the proposed approach, and areas for improvement.

\subsection{Implications of the Findings}
The proposed framework demonstrates superior performance in clustering quality, as evidenced by higher Silhouette Scores and lower Davies-Bouldin Index values compared to traditional methods. This indicates that the influence space, constructed from feature importance scores, provides a more meaningful representation for grouping energy consumption patterns. Moreover, the integration of temporal and contextual dynamics enables the framework to capture evolving subgroup behaviors, offering valuable insights for energy providers. For instance, the identification of subgroup transitions can inform targeted energy-saving interventions, while anomaly detection highlights rare consumption patterns that may indicate inefficiencies or unusual behaviors.

\subsection{Strengths of the Framework}
One of the key strengths of the framework is its ability to combine interpretability with dynamic analysis. Traditional clustering methods often struggle with high-dimensional or noisy data, and their results lack intuitive explanations. In contrast, the influence-based representation not only improves robustness but also allows stakeholders to understand the driving factors behind each subgroup. The transition matrix further adds a temporal dimension, revealing stability and variability in subgroup behaviors over time, which is particularly useful for longitudinal studies and predictive energy management.

\subsection{Limitations and Challenges}
Despite its strengths, the framework has certain limitations that merit discussion:
\begin{itemize}
    \item \textbf{Dependence on Predictive Model Quality:} The accuracy of the influence space depends on the predictive model used to compute feature importance scores. Poor model performance may result in unreliable influence scores, affecting the quality of clustering.
    \item \textbf{Scalability Issues:} The computational cost of generating influence scores, particularly for large datasets or complex models, may pose challenges in real-time applications.
    \item \textbf{Simplified Temporal Modeling:} While the transition matrix captures temporal dynamics, it does not explicitly model non-linear or long-term dependencies, which could limit its effectiveness in highly dynamic environments.
\end{itemize}

\subsection{Areas for Improvement}
Future work could address the above challenges by exploring the following enhancements:
\begin{itemize}
    \item Integrating ensemble-based XML methods to improve the robustness of the influence space, especially in scenarios with noisy or heterogeneous data.
    \item Developing efficient algorithms to accelerate the computation of influence scores, enabling scalability to larger datasets and real-time applications.
    \item Incorporating advanced temporal models, such as transformers or recurrent neural networks, to better capture complex dependencies in subgroup transitions.
    \item Extending the framework to unsupervised or semi-supervised learning tasks, enabling its application to datasets with limited labeled data.
\end{itemize}

\subsection{Broader Implications}
The proposed framework has potential applications beyond energy consumption analysis. Its ability to provide interpretable, dynamic clustering solutions can be valuable in domains such as healthcare, where evolving patient groups can inform personalized treatment strategies, or in financial analytics, where market dynamics could be studied through subgroup transitions. By generalizing the framework to diverse contexts, it could contribute to the broader adoption of explainable AI in dynamic data analysis.

In summary, the proposed framework addresses critical gaps in energy consumption analysis by offering a novel, interpretable, and dynamic clustering solution. While it demonstrates significant strengths, further advancements in computational efficiency, temporal modeling, and generalizability will enhance its utility and impact across domains.
\section{Conclusions and Future Work}
\label{sec:con}
In this paper, we proposed a novel \textit{Dynamic Influence-Based Clustering Framework} for energy consumption analysis. By leveraging explainable machine learning (XML) methods, the framework transforms energy data into an interpretable influence space, where feature importance scores provide a robust and meaningful basis for clustering. The framework integrates temporal and contextual dynamics to capture subgroup transitions and detect anomalies, addressing key challenges in analyzing energy consumption patterns. Experimental evaluations on real-world datasets demonstrated the effectiveness of the proposed framework, outperforming traditional clustering methods in terms of interpretability, clustering quality, and anomaly detection accuracy.

The key contributions of this work are summarized as follows:
\begin{itemize}
    \item We introduced an influence-based clustering approach that enhances subgroup interpretability by utilizing feature importance scores derived from XML methods.
    \item We developed a dynamic clustering mechanism that incorporates temporal and contextual dependencies, enabling the analysis of evolving subgroups in energy consumption.
    \item We proposed a transition-based anomaly detection method that identifies rare or unexpected patterns, providing actionable insights for energy management.
\end{itemize}

Despite the promising results, this work opens up several directions for future research. First, the proposed framework could be extended to handle multi-variate influence spaces derived from ensemble models, potentially improving robustness in scenarios with high-dimensional and heterogeneous data. Second, the current implementation focuses on supervised XML methods; future studies could explore unsupervised or semi-supervised techniques to generalize the approach to tasks where labeled data is limited. Third, integrating advanced temporal models, such as transformers or recurrent neural networks, could further enhance the ability to capture complex temporal dependencies in subgroup transitions. Finally, applying the framework to broader domains, such as financial analytics or healthcare data, could validate its generalizability and uncover new use cases.

In conclusion, this study contributes a novel, interpretable, and dynamic framework for energy consumption analysis, addressing critical gaps in existing methods. We believe the proposed approach has significant potential to inform energy management practices and inspire further innovations in dynamic data clustering and explainable AI.

\bibliographystyle{elsarticle-harv}
\bibliography{references}

\end{document}
